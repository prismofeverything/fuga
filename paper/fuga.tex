\title{Generating Fugal Sequences with Clojure, Occam and Reconstructability Analysis}
\author{Ryan Spangler}
\date{\today}

\documentclass[11pt]{article}

\usepackage{commath}
\usepackage{graphicx}
\usepackage{listings}
\usepackage{amsfonts}

% python highlighting ----------
\usepackage{color}
\usepackage{listings}
\usepackage{textcomp}
\usepackage{setspace}
\usepackage{hyperref}
%\usepackage{palatino}

%\doublespacing

\setcounter{secnumdepth}{0}

\begin{document}
\maketitle

\section{Introduction}

I have long suspected that Bach's fugues have a high degree of internal relatedness, though I have never had the tools to quantify that until now.  This project is a journey to define just how internally related Bach's fugues are, using the means at my disposal (namely Occam and Clojure, along with the theory behind RA).  It is an attempt to not just analyze, but also to generate new fugues using the form of old ones.  Given the scores of Bach's fugues from both books of his Well-Tempered Clavier (48 in all) as a raw collection of midi events, I tease apart the separate lines into sequences of notes and use these sequences to form predictions about what the next note is, given a series of previous notes.  Applying these predictions iteratively to a seed note generates new sequences of notes, which can then be layered and played back.  Two supporting Clojure libraries have sprung from this project, \href{http://github.com/prismofeverything/occam}{occam} (the clojure version!) and \href{http://github.com/prismofeverything/fuga}{fuga}.  These are open source projects anyone can use to analyze midi files and generate fugues of their own, which I discuss in the next section.  

Bach fugues are each unique, but have a similar overall structural philosophy.  They each start with a single voice which lays out a theme, and is ultimately joined by a series of voices each performing variations on the original.  Each line gets superimposed over the others, and part of the challenge of writing a fugue is tying these lines together in a meaningful and harmonious way (of which Bach is the generally acknowledged Khan of this arena).  In order to predict sequences, the common structures and gestures that thematically tie the fugue together have to be extracted and applied on new sequences.  Part of the symmetry of the fugal lines is that the theme is repeated, but in a different key or mode.  In this way, the information in contained in a relative, rather than an absolute way, and the analysis of the notes data has to reflect this relativity.

\section{Methods}

The process of generating new note sequences given a set of existing note sequences takes many steps.  First, the note files are parsed and the notes are extracted into a simple key based data format containing a field for the tone, the begin time and the end time.  In this form many notes could be happening at once, representing the separate voices of the fugue playing in polyphony.  In addition, the onset times are not exactly simultaneous, but fluctuate around a mean with epsilon less than 10 time steps.  So all notes are grouped into simultaneous occurances and these groups are sorted by their mutual onset.  Then, separate lines are drawn out of these groups by stepping through each coinciding time step and finding a previous and subsequent note from the previous and subsequent group for each note in that time step.  This step is not perfect and note jumps of more than 10 are excluded (arbitrarily, but better to assume too little!)  This reduces the series of notes into a handful of chains.  

Once the chains are in hand, they can be reduced to a series of snapshots of fixed length relative gestures.  This length is variable, so that a variety of lengths can be explored for predictionary power.  One of the main uses of Occam in this project was to discover how many previous notes should be taken into account to predict the next note.  I explored up to 5, where the defining the data would require two more notes than I wanted to use when doing the prediction.  Given a chain of notes [61 63 66 63 61 64 68] (where 60 is middle C) this can be pivoted on the second to last note (in this case the 64) and all of the previous notes can be masked relative to that one, resulting in [-3 -1 +2 -1 -3 0 +4].  The values leading up to the 0 are the previous notes in the chain, the 0 represents the value of the note we will be making the prediction for, and the last value is the dependent variable, the relative shift of the note given what notes came before it.  That way 7 notes can be used as a case for 5 previous notes predicting the subsequenct note given the current one.

All the chains are reduced to rows in the above relative format by passing a window over them, taking each seven values one after the other.  This is written out to Occam format and fed into the Occam web console.  In the above case, the independent variables are represented as ABCDE and the dependent variable is Z.  We don't need to the pivot note itself because it is always zero by definition:  all other notes are considered relative to it for this row.  This is the results matrix for five previous predicting notes A-E:

\begin{center}
\includegraphics[scale=0.45]{five.png}
\end{center}

From this table, the best model by information is CZ:DZ:EZ!  This is the point just before alpha flips to 1.0 from 0.0.  Also, this precedes a huge jump in dDF, which strikes a balance between simplicity and power.  Which means that only three previous notes need to be tracked to reasonably predict the next note, not all five.  

I take this to heart and run a fit on a series of three note predictions AZ:BZ:CZ.  The fit table is too large to print here, but can be found in the \href{http://github.com/prismofeverything/fuga}{fuga} repository under occam/fit/*.  This shows all the combinations of probabilities for each independent variable.  These all have a cardinality somewhere between 16 and 30, depending on how many relative jumps were really observed.  The dependent variable for this run has 15 observed states.  

The next step is parsing these fit tables into trees whose path for each row is given by the states of all the independent variables and whose value at that node is all of the possible outcomes and their relative probabilities (this is taken straight from the Occam output for this number of relative previous notes).  If nothing is found for that path, a shorter path is used on a similar tree created from data with only two previous notes (and therefore TWO independent variables).  If this again is not found, a third tree built on the data for a single previous note is used, which is always found.  Then, given a seed path (which can even be a single note), this process can be iterated, casting the three previous notes relative to the latest note and using this as a path into the probability tree, where the resulting state is found from choosing from the possible outcomes weighted by their relative probabilities.  This gives an arbitrarily long sequence of notes, which can then be recast into note events and passed on to the midi output player.  

Another consideration is for the note durations.  Given only the tones this would generate lines of all the same duration and would lose much of the interesting dynamics of the line.  Taking this into account the above process is repeated in its entirety for durations as well as tones, so there are two parallel paths of data, each representing a distinct DV, that are merged in the ultimate fugue generator.  

\section{Results}

The results cannot be entirely expressed in this paper, by nature of the medium.  They are notes which signify sounds and music, and must be heard and compared to the original fugues to make some kind of correspondence between them.  The instructions to run and hear the generated layered note sequences are given in the code repository for \href{http://github.com/prismofeverything/fuga}{fuga}.  Until papers allow auditory evocations, this will have to do.  

That said, the results sound plausibly fugue-like, if missing some of the ineffable binding and direction of the originals.  Recurring themes are evident, which are results of sharing common pathways of gestural unfolding.  Yet these lines are not mere repetitions, they evoke yet do not recreate the previous lines.  Some measure of the structure of the fugues is embedded in the possibility trees derived from Occam's analysis of the predictive power of the distinct signatures of the previous note chains.  

Some notable shortcomings are that they make no attempt to play in key, the lines are simply layered over one another and are not generated with respect to one another (which certainly contributes to their lack of tonality) and there is no global coherency in rhythm or pitch, only local coherency.  Clearly, there is more work to be done if the goal is to faithfully recreate the process of fugue generation, but there is some degree of success in the ability to generate endless sequences of fugue-inspired and fugue-resembling notes.

\section{Discussion}

Bach's fugues are complex.  The relations are far more elaborate than simple relative interval lines as were assumed in this study.  In order to truly capture the structure in a way that could generate plausible fugues, with all the richness and density of form and relation that lie embedded in the flat notation of the score, would take a great deal more work and subtlety than applied here.  

What would such a system look like?  It would have to have a deeper sense of the structure of the theme beyond simply relative intervals along a line.  Each line weaves in and out of each of the others, preserving both internal and external coherency.  How can this kind of omni-balancing be perceived and then applied?  An assumption of this whole approach is that these things are in a way interchangeable.  The perception of a event is expressed in the generation of subsequent events.  A more accurate perception of something would necessarily lead to a more accurate generation, synthesis and application of similar or novel events.  Is there a difference ultimately between perception and action?  Can something be perceived passively, without affecting subsequent action?  Anything that is perceived ultimately has the potential to affect anything else.  The relations that can be drawn, the influence and effect any event has on any subsequent action are never directly known, but can never be ruled out.  The network of known, perceived events is a global state that is constantly responding to every other known thing.  Nothing can be protected or isolated perfectly, everything has the potential to influence everything else.  Given that, the fugues cannot be accurately generated until they are accurately perceived.  This means having moving parts and structure in the analysis medium that coincide with the moving parts and structure that actually exist in the fugue.  

There is a point at which the perceiving system, the method of analysis, is only able to analyze what it is capable of quantifying and the nature of the algorithms behind that analysis.  That algorithm, whatever it is, is only effective in as much as it reveals what is actually happening, and not just what the algorithm itself is doing.  To construct a medium that given the scores of the Bach fugues in some form would capture the actual dynamics and relations that exist in the score itself, the richness of interrelation and mutual structure as the result of a common process, would be a serious yet worthy challenge.

One proposal is to model it as a network of connected nodes which continuously mediate a balance between the forces that drive them.  Many separate note events are aligned to identify with a single node of this network, and all of the differences in the context the nodes find themselves in during a period of generation express themselves as forces between these nodes, which compel them to find a global balance in a different configuration than before.  This would map to the themes in a fugue which recur yet never repeat exactly.  It would also allow the note generation to bend itself to the key and surrounding note context, guiding the way in which the nodal elements recur as unique note events.  

\section{Conclusion}

The goals set out in the beginning of this project were achieved, but this is really just the first step.  To be done is to make a fugue generator that actually generates fugues (and not simply quasi-fugues), with more coherency between the lines and an overall global integrity between the macro and micro scale structure.  In the meantime however, I will be listening to music constructed with Reconstructibility Analysis.  

\end{document}

